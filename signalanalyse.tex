% Dies ist Teil der Vorlesung Physik auf dem Computer, SS 2012,
% Axel Arnold, Universitaet Stuttgart.
% 
% Dieses Werk ist unter einer Creative Commons-Lizenz vom Typ
% Namensnennung-Weitergabe unter gleichen Bedingungen 3.0 Deutschland
% zugänglich. Um eine Kopie dieser Lizenz einzusehen, konsultieren Sie
% http://creativecommons.org/licenses/by-sa/3.0/de/ oder wenden Sie sich
% schriftlich an Creative Commons, 444 Castro Street, Suite 900, Mountain
% View, California, 94041, USA.

\chapter{Datenanalyse und Signalverarbeitung}

In diesem Kapitel geht es darum, was man mit einem gemessenen Signal
machen kann und muss. Ein gemessenes Signal kann dabei entweder
tatsächlich von einem Messgerät kommen oder aber das Ergebnis einer
Computersimulation sein. Zwei Fragen sind dabei vor allem wichtig:
welche Eigenschaften hat das Signal, und wie vertrauenswürdig sind die
Werte?

Um die Eigenschaften von Signalen zu untersuchen, ist die
kontinuierliche Fourieranalyse ein gutes Werkzeug, die das Signal vom
Zeit- in den Frequenzraum überträgt. So lassen sich zum Bespiel
charakteristische Frequenzen und damit Zeitskalen bestimmen. Außerdem
bietet der Übergang in der Frequenzraum analytisch viele Vorteile, die
sich auch auf dem Computer nutzen lassen. So werden zum Beispiel
langreichweitige Wechselwirkungen in Molekulardynamiksimulationen
meist im Frequenzraum berechnet.

Als weiteres Werkzeug werden wir Faltungen kennen lernen, die
erlauben, Signale nach bestimmten Frequenzen zu filtern oder aber aus
der (gemessenen) Antwort eines linearen Systems auf ein einfaches
Eingangssignal die Antwort auf beliebige Signale zu berechnen.

Sollen Signale mit dem Computer weiterverarbeitet werden, müssen diese
\emph{digitalisiert} werden, also in eine Reihe von Zahlen
übersetzt. Üblicherweise passiert dies dadurch, dass das Signal nur zu
äquidistanten Zeitpunkten ausgewertet, \emph{abgetastet} wird. Das
wirft die Frage auf, welche Funktionen dadurch überhaupt gut gemessen
werden können. Wie wir sehen werden, beschränkt diese Abtastung die
Frequenzen, die von einer digitalen Auswertung erfasst werden können.

Die meisten Signale sind außerdem, durch Messungenauigkeiten und
prinzipielle stochastische Prozesse, selbst \emph{stochastisch},
d.h. die Verteilung der Ergebnisse vieler Messungen ist
vorherbestimmbar, die einzelne Messung hingegen nicht. Trotzdem sind
Messungen oft korreliert, zum Beispiel weil eine Observable sich nur
kontinuierlich ändert.  Durch Korrelationsanalysen lässt sich
bestimmen, wann Messungen wirklich unabhängig sind. Dies gibt wiederum
Aufschluss über die Zeitskalen wichtiger Prozesse im System, ist aber
auch wichtig für eine korrekte Abschätzung des Messfehlers, womit sich
der letzte Abschnitt beschäftigt.

\section{Kontinuierliche Fouriertransformation}
\index{Fouriertransformation>kontinuierliche}

Für die Analyse zeitlich veränderlicher Signale besonders nützlich ist
die Fouriertransformation, die ein kontinuierliches Signal in den
Frequenzraum übersetzt. Dies gilt nicht nur für periodische Signale,
sondern zum Beispiel auch dann, wenn die Antwort eines Systems auf ein
komplexes Eingangssignal gefragt ist. Der tiefere Grund dafür ist,
dass die Fouriertransformation Differential- und Integraloperatoren in
einfache algebraische Operationen übersetzt.

Betrachten wir nochmals die Fourierreihe im Interval $[-T/2,T/2)$
\begin{multline}
  f(t) = \sum_{n\in\ZZ}
  \left(\frac{\Delta\omega}{2\pi}\int_{-T/2}^{T/2}
    f(t)e^{-i n\Delta\omega t}\, dt\right)
  e^{i n\Delta\omega t}\\
  =
  \frac{1}{\sqrt{2\pi}}\sum_{n\in\ZZ}
  \left(\frac{1}{\sqrt{2\pi}}\int_{-T/2}^{T/2} f(t)e^{-i \omega t}\,dt\right)
  e^{i \omega t}\,\Delta\omega
\end{multline}
mit der Grundfrequenz $\Delta\omega=2\pi/T$ und
$\omega=n\Delta\omega$. Im Grenzwert $T\to\infty$ ergibt sich
\begin{equation}
  \label{eq:contfourierinverse}
  f(t) = \frac{1}{\sqrt{2\pi}}\int_{-\infty}^{\infty}
  \FT(f)(\omega)e^{i \omega t}\,d\omega
\end{equation}
mit
\begin{equation}
  \label{eq:contfourier}
  \FT(f)(\omega) =
  \frac{1}{\sqrt{2\pi}}\int_{-\infty}^{\infty} f(t)e^{-i \omega t}\, dt.
\end{equation}
Die \emph{kontinuierliche Fouriertransformation} $\FT$ ist das
Analogon der periodischen Fourierreihe, ist allerdings keine
Transformation in eine Reihe mehr, sondern eine Abbildung zwischen
Funktionen. Für $\FT$ gelten eine Menge sehr starker Aussagen, die wir
zum großen Teil in ähnlicher Art schon von der Fourierreihe kennen:
\begin{itemize}
\item $\FT(f)$ existiert, falls $f$ quadratintegrabel ist, und
  bildet $f$ auf eine quadratintegrable Funktion ab. Für solche
  Funktionen mit der zugehörigen Norm $\lVert f \rVert_2 =
  \int_{-\infty}^{\infty} \lvert f(t) \rvert^2\,dt$ gilt dann sogar
  die Isometrie
  (\emph{Parsevaltheorem}\index{Parsevaltheorem>kontinuierliches}):
  \begin{equation}
    \lVert \FT(f) \rVert_2 = \lVert f \rVert_2
  \end{equation}
\item $\FT$ ist linear, d.h. $\FT(f + \lambda g) = \FT{f} + \lambda
  \FT{g}$.
\item $\FT$ ist reziprok gegen Streckungen, d.h.
  \begin{equation}
    \label{eq:fourierreziprok}
    \FT[f(\alpha
    t)](\omega) = \frac{1}{\lvert \alpha \rvert}
    \FT(f)\left(\frac{\omega}{\alpha}\right).
  \end{equation}
  Wird also eine Funktion $\alpha$ immer stärker gestaucht, so wird
  ihre Transformierte immer weiter gestreckt.

  Entsprechend wird aus Zeitumkehr Frequenzumkehr: $\FT(f(-t))(\omega)
  = \FT(f)(-\omega)$.
\item $\FT$ ist invertierbar, die Umkehrfunktion $\FT^{-1}$ ist durch
  \eqref{eq:contfourierinverse} explizit gegeben. Offenbar ist auch
  die Umkehrung eine Isometrie, es gilt $\lVert \FT^{-1}(f)
  \rVert_2 =\lVert f \rVert_2$.
\item Weiter gilt $\FT(\FT(f(t))) = \FT(f(-t))$, und damit $\FT^4(f) =
  f$. Insbesondere ist auch $\FT^{-1} = \FT^3$.
\item Eine zeitliche Verschiebung wird zu einer Frequenzmodulation und
  umgekehrt:
  \begin{eqnarray}
    \label{eq:fouriertimeshift}
    \FT(f(t-t_0))(\omega) = e^{-i\omega t_0} \FT(f(t))(\omega)\\
    \label{eq:fourierfreqshift}
    \FT(e^{i\omega_0 t}f(t))(\omega) = \FT(f(t))(\omega-\omega_0).
  \end{eqnarray}
  Wird also ein niederfrequentes Signal (Radioprogramm) auf ein
  hochfrequentes Trägersignal aufmoduliert, verschiebt sich nur dessen
  Spektrum. 
\item Aus der Linearität folgt, dass stets gilt:
  $\FT(\overline{f})(\omega) = \overline{\FT(f)(\omega)}$.
\item Ist Funktion $f$ symmetrisch, also $f(t) = f(-t)$, so ist
  $\FT(f)(-\omega) = \FT(f)(\omega)$, also symmetrisch
\item Ist Funktion $f$ ungerade, also $f(t) = -f(-t)$, so ist
  $\FT(f)(-\omega) = -\FT(f)(\omega)$, also ungerade.
\item Ist Funktion $f$ reellwertig, also $f(t) = \overline{f(t)}$, so
  ist $\FT(f)(-\omega) = \overline{\FT(f)(\omega)}$, aber im
  allgemeinen komplexwertig!
\item Für die Fouriertransformierte der Ableitung gilt
  \begin{equation}
    \begin{split}
      \FT\left(\frac{d}{dt}f(t)\right)(\omega) &=
      \frac{1}{\sqrt{2\pi}}\int_{-\infty}^{\infty}
      \frac{df}{dt}(t)e^{-i
        \omega t}\, dt\\
      &= \frac{1}{\sqrt{2\pi}}\int_{-\infty}^{\infty}
      f(t)\frac{d}{dt}e^{-i \omega t}\, dt = i\omega
      \FT(f(t))(\omega).
    \end{split}
  \end{equation}
  Dies spielt eine wichtige Rolle beim Lösen von
  Differenzialgleichungen, weil diese in gewöhnliche algebraische
  Gleichungen übergehen.
\item Es gilt die Poissonsche Summenformel
  \begin{equation}
    \label{eq:poissonsumme}
    \sum_{k\in\ZZ}f(t_0 + k\, \delta)=
    \frac{\sqrt{2\pi}}{|\delta|}\sum_{n\in\ZZ}\FT(f)
    \left(\frac{2\pi n}{\delta}\right)
    \exp\left(i \frac{2\pi n}{\delta} t_0\right).
  \end{equation}
  Diese Gleichung beruht darauf, dass $\sum_{t\in\ZZ}f(\cdot + t\,
  \delta)$ eine $\delta$-periodische Funktion ist, die durch eine
  Fourierreihe dargestellt werden kann. Wegen
  \eqref{eq:fouriertimeshift} reicht es
  dabei o.B.d.A. $\delta=1$ zu
  betrachten:
  \begin{multline}
    \sum_{k\in\ZZ}f(t_0 + k) =
    \sum_{n\in\ZZ} \int_0^1 \sum_{k\in\ZZ}f(\tau + k)e^{-2\pi i
      n \tau}\, d\tau
    \, e^{2\pi i n t_0} =\\
    \sum_{n\in\ZZ} \int_{-\infty}^\infty f(\tau)e^{-2\pi i
      n \tau}\, d\tau
    \, e^{2\pi i n t_0}
    = \sqrt{2\pi}\sum_{n\in\ZZ}\FT(f)(2\pi n)\, e^{2\pi i n t_0}.
  \end{multline}

  Eine wichtige Anwendung der Poissonschen Summenformel ist die
  Summation schlecht konvergenter Reihen. Fällt die Funktion $f$ sehr
  langsam gegen unendlich ab, so fällt ihre Fouriertransformierte
  wegen der Reziprozität \eqref{eq:fourierreziprok} im allgemeinen
  schneller, so das aus einer langsam eine rasch konvergierende Reihe
  wird.
\end{itemize}

\subsection{Spezielle Fouriertransformierte}

Die Fouriertransformierte einer Gaußglocke ist
\begin{equation}
  \FT\left(\frac{1}{\sqrt{2\pi}}e^{-t^2/2}\right)
  =
  \frac{1}{2\pi}e^{-\omega^2/2}\int_{-\infty}^{\infty} e^{-(t-i\omega)^2/2}\, dt
  =
  \frac{1}{\sqrt{2\pi}}e^{-\omega^2/2}
\end{equation}
Die Gaussglocke ist also eine Eigenfunktion der Fouriertransformation
zum Eigenwert 1. Die Fouriertransformation hat tatsächlich sehr viel
mehr echt verschiedene Eigenfunktionen, die Familie der
Hermitefunktionen~\cite{pinsky02a}. Wegen der Isometrieeigenschaft kann die
Fouriertransformation aber nur Eigenwerte vom Betrag eins haben;
tatsächlich hat sie nur die vier Eigenwerte $\pm 1$ und $\pm i$, da
$\FT^4=1$. Daher ist jeder Eigenwert stark degeneriert, und der
Eigenraum zu jedem Eigenwert unendlichdimensional.

Die (formale) Fouriertransformierte der $\delta$-Funktion ist
\begin{equation}
  \label{eq:fourierdelta}
  \FT(\delta(t))(\omega)
  =
  \frac{1}{\sqrt{2\pi}}\int_{-\infty}^{\infty}
  \delta(t)e^{-i \omega t}\, dt = \frac{1}{\sqrt{2\pi}},
\end{equation}
also einfach die konstante Funktion $1/\sqrt{2\pi}$, die ebensowenig
wie die $\delta$-Funktion eine quadratintegrable Funktion
ist. Alternativ lässt sich die Beziehung aus der
Fouriertransformierten der Gaußglocke mit sinkender Varianz herleiten.

\subsection{Numerische kontinuierliche Fouriertransformation}

Um die Eigenschaften der kontinuierlichen Fouriertransformation
numerisch zu nutzen, machen wir den Grenzübergang $T\to\infty$
rückgängig und ziehen uns auf ein für die betrachtetete Funktion
hinreichend großes $T$ zurück, so dass $\int_{\lvert t\rvert>T}\lvert
f(t)\rvert^2\, dt$ hinreichend klein ist. Ist das Signal wie in der
Praxis endlich, so könnte $T$ zum Beispiel so groß sein, dass das
Signal vollständig abgedeckt ist. In jedem Fall ist das Signal eine
Funktion $f$, die wir nur an auf einem äquidistanten Gitter $t_k =
T\left(-\frac{1}{2} + \frac{k}{N}\right)$, $k=0(1)N-1$ kennen. Dann
ist mit $\omega_0=\frac{2\pi}{T}$
\begin{multline}
  \label{eq:dftcont}
  \FT(f)(\omega_0 n)\approx
  \frac{1}{\sqrt{2\pi}}\int_{-T/2}^{T/2} f(t)e^{-i n \omega_0 t}\, dt
  \approx
  \frac{1}{\sqrt{2\pi}}\sum_{k=0}^{N-1}
  f(t_k)
  e^{-i n\omega_0 T\left(-\frac{1}{2} + \frac{k}{N}\right)}\,
  \frac{T}{N}\\
  =   \frac{T}{N} \frac{e^{-2\pi i n /2}}{\sqrt{2\pi}}\sum_{k=0}^{N-1}
  f(t_k) e^{-2\pi i n k / N}\,
  =
  \frac{T}{N}\frac{(-1)^n}{\sqrt{2\pi}} \text{DFT}(f(t_k))_n,
\end{multline}
wobei DFT die diskrete Fouriertransformation aus \eqref{eq:dft}
bezeichnet.  Die Koeffizienten der DFT sind periodisch, daher auch die
eben Näherung $\FT(f)$. Tatsächlich sollten die Koeffizienten aber nur
als Frequenzen im Interval $[-\omega_0 N/2,\omega_0 N/2]$
interpretiert werden, alle Frequenzen außerhalb dieses Intervals
sollten als Null angesehen werden. Der Grund dafür ist das
Abtasttheorem, dass im folgenden Abschnitt besprochen wird. Dieses
besagt, dass bei einem Zeitschritt $\Delta=T/N$ nur Kreisfrequenzen
bis $\omega_0 N/2 = \pi/\Delta$ eindeutig gemessen werden
können. Daher ist die Beschränkung auf das innerste Interval die
natürlich Interpretation.

Die bekannten schnellen FFT-Routinen lassen sich also
auch für die numerische Bearbeitung der kontinuierlichen
Fouriertransformation nutzen. Auf diese Weise ist
z.B. Abbildung~\ref{fig:fourierfaltung} entstanden.

\subsection{\keyword{Abtasttheorem}}

In der Praxis sind Signale meist als diskrete Werte an (endlich
vielen) äquidistanten Stellen beziehungsweise Zeitpunkten gegeben, zum
Beispiel, weil ein Messgerät Daten in regelmäßigen Abständen liefert.
Eine wichtige Frage ist, wie gut man das reale Signal und sein
Frequenzspektrum aus den diskreten Datenpunkten rekonstruieren kann.

Hierzu betrachten wir zunächst ein \emph{bandbreitenbeschränktes
  Signal} $f$, d.h. ein Signal, dessen Fouriertransformierte einen
kompakten Träger $[-\omega_0,\omega_0]$ hat. Dann besagt die
Poissonsche Summenformel~\eqref{eq:poissonsumme}, dass für
$\omega\in[-\omega_0,\omega_0]$
\begin{equation}
  \FT(f)(\omega) =
  \sum_{k\in\ZZ}\FT(e^{-i\omega \cdot} f)\left(2k\omega_0\right)
  =
  \frac{1}{\sqrt{2\pi}}
  \sum_{n\in\ZZ} e^{-i \omega n \Delta}f\left(n \Delta\right)
\end{equation}
mit $\Delta=\pi/\omega_0$. Die Fouriertransformierte eines
bandbreitenbeschränkten Signals lässt sich also \emph{exakt} nur aus
den Funktionswerten an den äquidistanten diskreten Stellen mit
Abtastrate $\Delta$ berechnen. Dadurch kann natürlich auch die
Funktion exakt aus ihrer Fouriertransformierten rekonstruiert werden:
\begin{equation}
  \label{eq:ftreconst}
  f(t)
  = \frac{\Delta}{2\pi}\sum_{n\in\ZZ} f\left(n \Delta\right)
  \int_{-\omega_0}^{\omega_0} e^{-i \omega (t - n\Delta)} \,d\omega
  = \frac{\Delta}{\pi}\sum_{n\in\ZZ} f\left(n \Delta\right)
  \frac{\sin(\omega_0(t-n\Delta))}{t-n\Delta}.
\end{equation}

Ist nun umgekehrt eine Signal $f(n\Delta)$ an äquidistanten diskreten
Stellen gegeben, so lässt sich diesem umgekehrt gemäß
\eqref{eq:ftreconst} ein kontinuierliches Signal zuordnen, dessen
Fouriertransformierte nur in $[-\omega_0,\omega_0]$ nicht
verschwindet, wobei $\omega_0 = \pi/\Delta$. Das bedeutet, dass bei
Abtastrate $\Delta$ nur Frequenzen bis zur Nyquist-Frequenz
$f_\text{Nyquist}=\frac{1}{2\Delta}$ eindeutig abgetastet werden
können, ähnlich wie im periodischen Fall.

\section{Faltungen}
\index{Faltung}

Die \emph{Faltung} der quadratintegrablen Funktionen $f$ und $g$ ist
definiert als
\begin{equation}
  (f \star g)(t) := \int_{-\infty}^{\infty} f(t')g(t-t')\,dt'.
\end{equation}
Das negative Vorzeichen von $t'$ in der zweiten Funktion sorgt
dafür, dass die Faltung kommutativ ist. Weitere Eigenschaften der
Faltung sind Linearität in den Komponenten und sogar
Assoziativität. Die Faltung verhält sich also so ähnlich wie die
klassische Multiplikation und wird daher mit dem Zeichen $\star$
bezeichnet.  Vereinfacht gesagt, deponiert die Faltung an jedem Punkt
$t$ die Funktion $f$, skaliert mit $g(\cdot - t) $. Daher ist z.B.
\begin{equation}
  (\delta(\cdot - t_0) \star g)(t) = g(t - t_0).
\end{equation}
\index{Glättung}
Wird nun statt der unendlich dünnen $\delta$-Funktion zum Beispiel
eine Gaußglocke gewählt, so wird die Funktion $g$ verschmiert
bzw. geglättet, siehe Abbildung~\ref{fig:fourierfaltung}.

\begin{figure}
  \centering
  \includegraphics[width=\textwidth]{plots/fouriertrafo}
  \caption{Oben links: Summe zweier Schwingungen $\cos(x) +
    0,25\sin(7,5 x)$ (schwarze Linie), die mit einer Gaußglocke (blau
    gepunktet) gefaltet wird. Das Ergebnis (rote dicke Linie) ist die
    quasi ungestörte langsame Schwingung ohne die höherfrequente
    Schwingung, die weggeglättet wurde. Oben rechts:
    Fouriertransformierte der Funktionen. Klar sichtbar ist die
    hochfrequente Störung mit einer Frequenz von 7,5, die im
    gefilterten Signal fehlt. In der unteren Reihe wurde eine
    frequenzverschobene Gaußglocke benutzt, um statt der langsamen die
    schnelle Schwingung zu filtern; die Farbcodierung ist wie
    oben. Gemäß \eqref{eq:fourierfreqshift} bewirkt die
    Frequenzverschiebung eine Modulation der Gaußfunktion.}
  \label{fig:fourierfaltung}
\end{figure}

Die Fouriertransformierte der Faltung zweier Funktionen ist
\begin{equation}
  \label{eq:fourierfaltung}
  \begin{split}
    \FT(f\star g)(\omega) &=
    \frac{1}{\sqrt{2\pi}}\int_{-\infty}^{\infty}
    \int_{-\infty}^{\infty} f(t')g(t-t')\,dt' e^{-i \omega t}\,
    dt\\
    &= \frac{1}{\sqrt{2\pi}}\int_{-\infty}^{\infty}
    \int_{-\infty}^{\infty} f(t') g(t) e^{-i \omega (t+t')}\, dt\,
    dt'\\
    &= \frac{1}{\sqrt{2\pi}} \int_{-\infty}^{\infty} f(t')e^{-i
      \omega t'}\, dt' \int_{-\infty}^{\infty} g(t) e^{-i \omega
      t}\, dt = \sqrt{2\pi}\FT(f)(\omega)\FT(g)(\omega).
  \end{split}
\end{equation}
Die Faltung geht also in eine punktweise Multiplikation über. Im
Fourierraum lässt sich also sehr viel schneller falten, als im
Realraum, wo ja für jeden Punkt ein Integral zu lösen ist.

\subsection{Filter}
\index{Filter}

Außerdem lässt \eqref{eq:fourierfaltung} noch eine weitere
Interpretation der Faltung der Funktion $g$ mit der Funktion $f$ zu:
ist $f$ bzw. $\FT(f)$ reellwertig und symmetrisch, so werden die
einzelnen Frequenzkomponenten der Funktion $g$ mit den
Frequenzanteilen von $f$ gestreckt bzw. gestaucht, $g$ also
frequenzgefiltert.  Durch Wahl einer symmetrischen und reellwertigen
Fouriertransformierten und Rücktransformation lassen sich also
beliebige Frequenzfilter realisieren; in der Praxis wird natürlich
direkt im Fourierraum gefiltert. Später lernen wir die numerische
diskrete Fouriertransformation kennen, die nicht zuletzt wegen dieser
Filtereigenschaften so wichtig ist. Abbildung~\ref{fig:fourierfaltung}
illustriert einen (gaußschen) Tiefpassfilter und einen Bandfilter, die
nur bestimmte Frequenzen passieren lassen. Daher wird beim
Tiefpassfilter die aufgeprägte hochfrequente Schwingung unterdrückt,
beim Hochpassfilter hingegen die an sich dominante langsame Schwingung.

\subsection{Antwort zeitinvarianter linearer Systeme}
\index{zeitinvariante lineare Systeme}

Eine weitere wichtige Anwendung der Faltung ist die Bestimmung der
Antwort eines zeitinvarianten linearen Systems auf ein beliebiges
Eingangssignal. Einfach zu messen ist typischerweise die Antwort
$A_\theta(t)$ des Systems auf einen Einschaltvorgang, also ein
Eingangssignal der Form
\begin{equation}
  \theta(t) =
  \begin{cases}
    0 & \text{für}\; t < 0\\
    1 & \text{für}\; t \ge 0.
  \end{cases}
\end{equation}
Um daraus die Antwort auf ein beliebiges Eingangssignal zu bestimmen,
schreiben wir das Eingangssignal $f$ als $f = f \star \delta$. Wegen
der Linearität der Faltung und der Systemantwort ist die Antwort auf
das Signal $f$ gegeben durch die Faltung $A_f(t) = f \star A_\delta$
mit der Antwort auf einen $\delta$-Impuls. Diese Antwort wiederum
lässt sich aus der Sprungantwort durch einfach Ableitung erhalten, was
mit Hilfe der Fouriertransformierten sehr bequem zu berechnen ist:
\begin{equation}
  A_f(t) = f \star A_\delta = f \star
  \frac{d}{dt} A_\theta = \sqrt{2\pi} \FT^{-1}\bigl( i\omega \FT(f) \FT(A_\theta)\bigr).
\end{equation}

\section{Kreuz- und Autokorrelation}
\index{Korrelationsanalyse}

Bis jetzt haben wir uns mit der Verarbeitung idealer Signale
beschäftigt, die zu einem gegebenen Zeitpunkt einen prinzipiell
eindeutig vorherbestimmten Wert haben.  Reale Signale sind aber oft
verrauscht, entweder durch Messungenauigkeiten, Bauteiltoleranzen oder
prinzipielle stochastische Prozesse.  Trotzdem möchte man oft wissen,
ob zwei gemessene Signale von einander abhängig, \emph{korreliert},
sind. Zum Beispiel könnte man die Position eines Elektrons und seinen
Spin, die Menge der verkauften Eis- und Sonnencreme, oder auch die
Position eines Pendels zu zwei verschiedenen Zeitpunkten betrachten.
In den beiden letzteren Fällen werden diese im allgemeinen korreliert,
also abhängig sein. Allerdings gibt dies keinen Aufschluss über den
dahinterstehenden kausalen Mechanismus. Im Fall des Pendels rührt die
Korrelation daher, dass es sich in kurzer Zeit nicht beliebig weit von
seiner Ausgangsposition bewegen kann. Bei der Eis- und Sonnencreme
wird es vermutlich auch eine Korrelation geben, aber weder erzeugt
Eiscreme Sonnenbrand, noch macht Sonnencreme Lust auf Eis. Allerdings
haben wir Menschen nunmal bei strahlendem Sonnenschein mehr Lust auf
Eis, aber brauchen auch Sonnencreme.

Formal betrachten wir zunächst zwei Observablen $A$ und $B$. Diese
beiden heißen genau dann unkorreliert, falls die Mittelwerte
$\mean{A\cdot B} = \mean{A}\mean{B}$
bzw. $\mean{(A-\mean{A})(B-\mean{B})}=0$ erfüllen.  Im allgemeinen
wird man aber vielleich nicht erwarten, dass sich die Änderung einer
Observablen unmittelbar in einer anderen niederschlägt, sondern erst
nach einer Zeit $\tau$.  Man betrachtet daher die Korrelation zwischen
$A$ und $B$ mit einem zeitlichen Versatz von $\tau$, die
\emph{\keyword{Kreuzkorrelationsfunktion}}:
\begin{equation}
  \label{eq:crosscorr}
  C(A,B)(\tau) := \mean{A(0)B(\tau)},
\end{equation}
wobei die Signale $A$ und $B$ zeitinvariant sein sollen, so dass der
Zeitpunkt $t=0$ beliebig gewählt sein kann. Für große $\tau$
dekorrelieren die Signale, daher gilt $C(A,B)\to \mean{A}\mean{B}$ für
$\tau\to\infty$. Wird stattdessen die normierte Kreuzkorrelation
$C(A-\mean{A}, B-\mean{B})$ betrachtet, verschwindet dieses also im
Limit $\tau\to\infty$.

$\mean{\cdot}$ bezeichnet in der Physik üblicherweise den
Ensemblemittelwert, also den Mittelwert über alle möglichen
Realisationen des Experiments. Sind nun $A=A(t)$ und $B=B(t)$
zeitliche Messreihen eines zeitinvarianten Systems, so ermittelt man
die Mittelwerte üblicherweise als Zeitmittelwerte, also Integrale über
die Zeit:
\begin{equation}
  \label{eq:crosscorrtime}
  C(A,B)(\tau) = \mean{A(0)\cdot B(\tau)} \stackrel{!}{=}
  \lim_{T\to\infty}\frac{1}{T}\int_{-T/2}^{T/2} A(t)B(t+\tau)\,dt.
\end{equation}
Das Ausrufezeichen soll andeuteten, dass dies eine Annahme ist, denn
die Gleichheit gilt nur genau dann, wenn das System \emph{ergodisch}
ist, d.h., dass der Prozess bei einer unendlich langen zeitlichen
Messung alle möglichen Realisationen einmal besuchen wird. Diese
Annahme wird meist gemacht, obwohl die Ergodizität für die meisten
Systeme nicht bewiesen werden kann. Hinzu kommt, dass ja in der Praxis
niemals über beliebig lange Zeiträume gemittelt werden kann. Daher
können auch endliche, aber hohe Energiebarrieren zu einem
systematischen Fehler führen.

$\mean{A}$ bzw. $\mean{B}$ sind oft gar nicht genau bekannt, und
müssen numerisch durch (Zeit-)Mittelung der Daten bestimmt werden. Da
hier aber normalerweise dieselben Daten zugrunde gelegt werden, die
auch zu Berechnung der Kreuzkorrelation selber genutzt werden, sind
diese notwendigerweise korreliert, was zu kleinen Abweichungen in der
Kreuzkorrelationsfunktion führt. Im häufigen Fall, dass eine
Observable aus Symmetriegründen einen Mittelwert von Null haben muss,
sollte daher auf keinen Fall der numerische gemessene Mittelwert
abgezogen, "`um das Ergebnis zu verbessern"'. Diese übliche Praxis ist
falsch, da sie ja erzwingt, dass der letzte Datenpunkt der gemessenen
Kreuzkorrelationsfunktion notwendigerweise auf $\mean{A}\mean{B}$
abfällt, selbst wenn einfach nur das Messinterval zu kurz gewählt
wurde. Daher führt diese Methode zu einer Unterschätzung der
Langzeitkorrelationen!

Analog zu \eqref{eq:crosscorrtime} kann man die Kreuzkorrelation auch
für \emph{endliche} Signale definieren. Für zwei quadratintegrable
Signale $f$ und $g$ ist in Analogie die Kreuzkorrelationsfunktion
definiert als
\begin{equation}
  C(f,g)(\tau) = \int_{-\infty}^{\infty} f(t)g(t+\tau)\,dt,
\end{equation}
wobei wegen der Quadratintegrabilität auf die Normierung verzichtet
werden kann. Diese Kreuzkorrelation misst keine Korrelationen im
stochastischen Sinne mehr, weil das Integral nun keine Zeitmittelung
mehr darstellt. Stattdessen ist $C(f,g)(\tau)$ in diesem Fall ein Maß
dafür, wie sehr sich die Signale $f$ und $g$ mit einem Zeitversatz
$\tau$ im Verlauf ähneln. Diese Form der Kreuzkorrelation ähnelt einer
Faltung sehr. Tatsächlich ist
\begin{equation}
  C(f,g)(\tau) = \bigl(f \star g(-\cdot)\bigr)(-\tau) =
  \bigl(f(-\cdot) \star g\bigr)(\tau)
\end{equation}
und kann damit effizient im Frequenzraum bestimmt werden:
\begin{equation}
  \label{eq:crosscorrfft}
  C(f,g) = \sqrt{2\pi}\FT^{-1}
  \bigl(\FT(f)(-\omega)\FT(g)(\omega)\bigr).
\end{equation}

Kommen wir nun zu unserem Ausgangsproblem mit zwei stochastischen,
zeitinvarianten Variablen $A$ und $B$ und dem Zeitmittel zurück. Wie
üblich nehmen wir an, dass $A$ und $B$ an endlich vielen, diskreten
Zeitpunkten $tk\Delta$, $k=0(1)n-1$, gemessen wurden. Die Messung
erstreckt sich also über den Zeitraum $[0,T]$ mit $T=n\Delta$. Dann
ist eine Näherung für die Kreuzkorrelation von $A$ und $B$ gegeben
durch
\begin{equation}
  \label{eq:crosscorrnum}
  C(A,B)(k\Delta) = \mean{A(0)\cdot B(k\Delta)}
  \approx
  \frac{1}{N-k}\sum_{l=0}^{N-k}A(l\Delta)B\left((l+k)\Delta\right)
  \quad\text{für}\; k=0(1)K-1.
\end{equation}
Die unterschiedliche Gewichtung $1/(N-k)$ ergibt sich durch die
unterschiedliche Anzahl an Messungen für den Versatz $k$. Für große
Versätze ist die Anzahl der Messungen sehr klein (für $k=N-1$ nur noch
eine), daher muss $k \ll N$ sein. Form \eqref{eq:crosscorrnum} ist
numerisch allerdings nicht sehr effizient auszuwerten, da im
allgemeinen $2NK$ viele Operationen benötigt werden, und $K$ meist
einige hundert Datenpunkte beträgt.  Daher liegt es nahe, auch hier
die FFT zur Beschleunigung einzusetzen. \eqref{eq:crosscorrfft} und
\eqref{eq:dftcont} zusammen ergeben für gleich lange Messreihen $A$ und
$B$:
\begin{multline}
  \label{eq:crosscorrnum}
  C(A,B)(k\Delta)
  \approx
  \sqrt{2\pi}\FT^{-1}
  \bigl(\FT(f)(-\omega)\FT(g)(\omega)\bigr)(k\Delta)\\
  \approx
  \frac{1}{N}\,\text{iDFT}
  \bigl(\overbar{\text{DFT}(A)(n)}\text{DFT}(B)(n)\bigr)(k).
\end{multline}
Man sollte dabei beachten, dass durch die Benutzung der DFT das Signal
implizit mit Periode $N\Delta$ periodisiert wird. Daher sollte
$C(A,B)(k\Delta)$ nur für $k\ll N/2$ interpretiert werden. Die
Berechnung der 2 DFTs sowie der inversen DFT braucht etwa $6 N\log N$
Operationen, was normalerweise wesentlich weniger als die $2NK$ der
direkten Berechnung ist.

In Python sieht die Berechnung der Kreuzkorrelation so aus:
\begin{lstlisting}
import numpy
import numpy.fft as fft

def kreuzkorrelation(A, B):
    ftA = fft.fft(A).conj()
    ftB = fft.fft(B)
    return numpy.real(fft.ifft(ftA*ftB))/A.shape[0]
\end{lstlisting}

\begin{figure}
  \centering
  \includegraphics[width=0.5\textwidth]{plots/v_ac}
  \caption{Geschwindigkeitsautokorrelationsfunktion einer temperierten
    Lennard-Jones-Flüs\-sig\-ke\-it mit niedriger Dichte (rot
    gepunktet) und hoher Dichte (blau durchgezogen). Für die niedrige
    Dichte ist die Autokorrelationsfunktion im wesentlichen
    exponentiell abfallend, mit einer Korrelationszeit $\tau_c=0.17$,
    die durch die Kopplungszeit des Thermostaten bestimmt ist.}
  \label{fig:vac}
\end{figure}

\subsection{Autokorrelationsfunktion}

Im Spezialfall $A=B$ spricht man von der
\emph{\keyword{Autokorrelationsfunktion}}. Diese ist offenbar
symmetrisch, daher sind nur Zeitversätze $\tau\ge 0$ von
Interesse. Die Autokorrelationsfunktion misst gewissermaßen, wie lange
es dauert, bis die Observable nicht mehr von seinem vorherigen Wert
abhängt, wann es diesen sozusagen "`vergisst"'. In dem häufigen Fall,
dass die Autokorrelationsfunktion zunächst expontentiell abfällt,
lässt sich dem Gedächtnis eine Zeitkonstante $\tau_c$, die
\emph{\keyword{Korrelationszeit}}, zuordnen. Diese kann man entweder
durch einen geeigneten Funktionsfit bestimmen, oder aber durch
Integration aus
\begin{equation}
  \label{eq:tauc}
  \int_{0}^{\infty} C(A,A)(\tau)\,d\tau = \int_{0}^{\infty}
  C(A,A)(0)e^{-\tau/\tau_c}\,d\tau = \tau_c\,C(A,A)(0).
\end{equation}

Abbildung~\ref{fig:vac} zeigt die Geschwindigkeitsautokorrelation
$C(v,v)$ einer temperierten Lennard-Jones-Flüs\-sig\-ke\-it bei zwei
verschiedenen Dichten. Die Temperierung wird dabei mit Hilfe eines
Thermostaten erreicht, der die Teilchen stochastisch an ein Wärmebad
koppelt. Dadurch dekorreliert die Geschwindigkeit eines Teilchens in
einer Korrelationszeit von etwa $1/6$.

$C(v,v)$ wird häufig dazu benutzt, um die Diffusionskonstante $D =
\int_0^{\infty} C(v,v)(\tau)\, d\tau$ des Systems zu bestimmen, die
also eng mit der Korrelationszeit des Thermostaten verwandt ist.  Bei
niedriger Dichte ist das System annähernd ein ideales Gas, und die
Teilchen dekorrelieren im wesentlichen nur durch den Einfluss des
Thermostaten, der durch Zufallskräfte wirkt, daher ist $C(v,v)$
tatsächlich gut exponentiell abfallend.  Formel~\eqref{eq:tauc}
bestimmt die Korrelationszeit mit guter Genauigkeit zu etwa
$\tau_c=0.17$, wie durch den Thermostaten zu erwarten. Im Falle der
dichteren Flüssigkeit hingegen kann diese Formel nicht angewendet
werden, da die Autokorrelation kein einfacher exponentieller Abfall
mehr ist, da auch Stoßprozesse eine wichtige Rolle spielen. Diese
führen zum Durchschwingen der Autokorrelationsfunktion.

\section{Messfehlerabschätzung}
\index{Messfehler}

In diesem letzten Abschnitt zur Datenanalyse geht es darum, wie der
der Messfehler bei der Messung des Erwartungswerts einer
stochastischen Observable abgeschätzt werden kann. Wir betrachten also
eine Messreihe $x_i$, $i=1(1)N$, die verschiedene Messungen einer
stochastischen Observablen $x$ darstellen. Der Erwartungswert dieser
Observablen lässt sich dann bekanntermaßen als
\begin{equation}
  \label{eq:mean}
  \mean{x} \approx \bar{x} := \frac{1}{N}\sum_{i=1}^N x_i
\end{equation}
abschätzen. Doch was ist nun der Fehler, den wir mit dieser Schätzung
machen? Dieser ist die zu erwartende quadratische Abweichung
\begin{equation}
  \mean{\left(\overline{x} - \mean{x}\right)^2} =
  \frac{1}{N^2}\sum_{i,j=1}^N \mean{x_ix_j} -
  \frac{2}{N}\sum_{i=1}^N \mean{x_i}\mean{x} + \mean{x}^2
  = \frac{2}{N^2}\sum_{i > j} \mean{x_ix_j}
  + \frac{1}{N}\mean{x^2} - \mean{x}^2
\end{equation}
An dieser Stelle nimmt man nun an, dass die Messungen paarweise
unabhängig sind, also, dass $\mean{x_ix_j} = \mean{x_i}\mean{x_j}$ für
$i\neq j$.  In der Praxis lässt sich das zum Beispiel durch Betrachten
der Autokorrelationsfunktion sicherstellen, in dem nur Messwerte mit
einem zeitlichen Abstand berücksichtigt werden, der sehr viel größer
als die Korrelationszeit ist. Unter der Annahme, dass die Messungen
$x_i$ alle paarweise unabhängig sind gilt also weiter
\begin{equation}
  \label{eq:esterr}
  \mean{(\overline{x} - \mean{x})^2}
  = \frac{N(N-1)}{N^2}\mean{x}^2 + \frac{1}{N}\mean{x^2} - \mean{x}^2
  = \frac{1}{N}\left(\mean{x^2} - \mean{x}^2\right)
  = \frac{1}{N}\sigma^2(x).
\end{equation}
Sind die Messungen unabhängig voneinander, ist der quadratische Fehler
von $N$ Messungen also gerade ein $N$-tel der Varianz
\begin{equation}
  \sigma^2(x) = \mean{\bigl(x-\mean{x}\bigr)^2}
  = \mean{x^2} - 2\mean{\mean{x}x}  + \mean{x}^2
  = \mean{x^2} - \mean{x}^2,
\end{equation}
die die erwartete quadratische Abweichung einer Messung vom Mittelwert
angibt. Sie ist eine Eigenschaft der zu messenden Observablen und daher
nicht von der Anzahl der Messungen abhängig.

Als Fehlerbalken wird üblicherweise die \emph{Standardabweichung} der
Messung $\overline{x}$, angegeben. Diese ist durch
\begin{equation}
  \sqrt{\mean{(\overline{x} - \mean{x})^2}}
  = \frac{1}{\sqrt{N}}\sigma(x)
\end{equation}
gegeben. Dies bedeutet, dass für eine Halbierung des Fehlerbalken
bereits viermal soviele Messungen durchgeführt werden müssen, und für
eine Größenordnung an Genauigkeit hundert Mal soviele. Besonders für
Computersimulationen ist das ein Problem, da die Rechenzeit im
allgemeinen proportional zur Anzahl der Messungen ist. Dauert also
eine Simulation eine Woche, was nicht ungewöhnlich ist, so würde eine
Messung mit einer Größenordnung mehr Genauigkeit fast zwei Jahre in
Anspruch nehmen!

Zur Berechnung des Fehlers benötigen wir noch eine Schätzung der
Varianz, die im allgemeinen ebensowenig wie der Mittelwert bekannt ist
und geschätzt werden muss. Dazu ersetzt man die Mittelwerte in
$\mean{x^2} - \mean{x}^2$ durch den Schätzer \eqref{eq:mean}. Für
diesen Ausdruck gilt:
\begin{equation}
  \label{eq:varest}
  \mean{\overline{x^2} - \bar{x}^2}
  = \mean{x^2} - 
  \frac{1}{N^2}\sum_{i=1}^n \mean{x_i^2}
  - \frac{1}{N^2}\sum_{i\neq j} \mean{x_ix_j}
  = \frac{N-1}{N}\left(\mean{x^2} - \mean{x}^2\right),
\end{equation}
wobei wir wieder annehmen, dass die Messungen unabhängig voneinander
sind. Das der Ausdruck auf der linken Seite selber kein guter
Schätzer ist, liegt also daran, dass für $\bar{x}$ zweimal derselbe
Datensatz benutzt wurde. Ein guter Schätzer für $\sigma^2(x)$ ergibt
sich aus der Umkehrung von \eqref{eq:varest}:
\begin{equation}
  \sigma^2(x) \approx \frac{N}{N-1}\left(\overline{x^2} - \bar{x}^2\right)
  = \frac{1}{N-1} \left(\sum_{i=1}^N x_i^2 - N\bar{x}^2\right).
\end{equation}

Auf dem Computer lassen sich also $\mean{x}$ und $\sigma^2(x)$ bequem in
einem Durchlauf der Daten abschätzen:
\begin{lstlisting}
sum  = 0
sum2 = 0
for v in x:
    sum  += v
    sum2 += v*v
mittel = sum/len(x)
sigma2 = (sum2 - len(x)*mittel**2)/(len(x)-1)
fehler = sqrt(sigma2/N)
\end{lstlisting}

In der Praxis wird oft einfach angenommen, dass die Messungen
unabhängig sind. Was passiert nun, wenn dies nicht der Fall ist?
Betrachten wir also eine Observable die o.B.d.A. Mittelwert $\mean{x}
= 0$ habe. Dann ist wie oben gezeigt
\begin{align}
  \mean{(\overline{x} - \mean{x})^2}
  = \frac{2}{N^2}\sum_{i > j} \mean{x_ix_j}
  + \frac{1}{N}\sigma^2(x).
\end{align}
Der tatsächliche Fehler ist also um $\frac{2}{N^2}\sum_{i > j}
\mean{x_ix_j}$ größer als man aufgrund der Varianz erwarten würde. In
die Abschätzung für die Varianz ging ebenfalls die Unabhängigkeit ein,
für diese gilt nun
\begin{align}
  \frac{N}{N-1}\mean{\overline{x^2} - \bar{x}^2}
  &= \frac{1}{N-1}\left(N\mean{x^2} - \mean{x^2}
    - \frac{1}{N}\sum_{i\neq j} \mean{x_ix_j}\right)\\
  &= \sigma^2(x) - \frac{2}{N(N-1)}\sum_{i> j} \mean{x_ix_j},
\end{align}
d.h., die Varianz wird zusätzlich noch um
$\frac{1}{(N-1)N^2}\sum_{i\neq j} \mean{x_ix_j}$
\emph{unterschätzt}. Insgesamt wird der quadratischen Fehler also
um ebenfalls
\begin{align}
  F&=\mean{(\overline{x} - \mean{x})^2} -
  \frac{1}{N}\frac{N}{N-1}\mean{\overline{x^2} - \bar{x}^2}\nonumber\\
  &=
  \left(\frac{2}{N^2} + \frac{2}{(N-1)N^2}\right)\sum_{i > j} \mean{x_ix_j}
  =
  \frac{2}{N(N-1)}\sum_{j=1}^{N}\sum_{n = 1}^{N-j} \mean{x_jx_{j+n}}
\end{align}
unterschätzt. Für eine Observable $x$, die in gleichmäßigen
Zeitabständen $\Delta$ gemessen wird, lässt sich dieser Ausdruck mit
Hilfe der Autokorrelationsfunktion $C(\tau) = C(x,x)(\tau)$
abschätzen:
\begin{equation}
  \label{eq:correrrest}
  F
  \approx
  \frac{2}{N(N-1)}\sum_{j=1}^{N}\int_{n = 0}^{N-j} C(n\Delta)
  \approx
  \frac{2}{N}\int_{n = 0}^{\infty} C(n\Delta)
  =
  \frac{2}{N}\frac{\tau_c}{\Delta} C(0)
  =
  \frac{2\tau_c}{\Delta}\frac{\sigma^2(x)}{N},
\end{equation}
wobei $\tau_c$ die Korrelationszeit gemäß \eqref{eq:tauc} ist und
angenommen wurde, dass $\tau_c\ll N\Delta$, so dass die Erweiterung
des Integrals bis unendlich keinen nennenswerten Beitrag mehr liefert.
Verglichen mit \eqref{eq:esterr} erhält man mit korrelierten Daten
eine Abschätzung für den Fehler, die um einen Faktor
$1-2\tau_C/\Delta$ zu klein ist, was sich durch noch so gute Statistik
nicht ausgleichen lässt! Daher lassen sich ohne Kenntnis der
Korrelationszeit der Daten keine verlässlichen Aussagen über die Güte
der Daten machen. Nur, wenn $\Delta\gg\tau_c$, ist der Fehler durch
korrelierte Daten vernachlässigbar.

\begin{figure}
  \centering
  \includegraphics[width=0.5\textwidth]{plots/error}
  \caption{Geschätzte, scheinbare Varianz eines dünnen temperierten
    Lennard-Jones-Systems als Funktion der gewählten Schrittweite
    $\Delta$.  Die schwarze Linie zeigt die Abschätzung der
    scheinbaren Varianz gemäß \eqref{eq:correrrest} als
    $(1-2\tau_C/(\Delta N))\sigma^2(x)$, was gut zu den gemessenen
    Daten passt.}
  \label{fig:error}
\end{figure}

Abbildung~\ref{fig:error} zeigt die scheinbare Varianz des weniger
dichten thermalisierten Lennard-Jones-Systems als Funktion der
gewählten Schrittweite. Wie gezeigt ist die Korrelationszeit dieses
Systems $\approx 0,17$, so dass bei rascheren Messungen eine
Unterschätzung der Varianz und auch des Fehlers zu erwarten ist, was
von der Abbildung bestätigt wird.

%%% Local Variables: 
%%% mode: latex
%%% TeX-master: "padc"
%%% TeX-PDF-mode: t
%%% End: 
