% Dies ist Teil der Vorlesung Physik auf dem Computer, SS 2012,
% Axel Arnold, Universitaet Stuttgart.
% 
% Dieses Werk ist unter einer Creative Commons-Lizenz vom Typ
% Namensnennung-Weitergabe unter gleichen Bedingungen 3.0 Deutschland
% zugänglich. Um eine Kopie dieser Lizenz einzusehen, konsultieren Sie
% http://creativecommons.org/licenses/by-sa/3.0/de/ oder wenden Sie sich
% schriftlich an Creative Commons, 444 Castro Street, Suite 900, Mountain
% View, California, 94041, USA.

\chapter{Einleitung}

In dieser Vorlesung geht es darum, wie der Computer in der modernen
Physik eingesetzt wird, um neue Erkenntnisse zu gewinnen. Klassisch
war die Physik ein Zusammenspiel aus Experiment und Theorie. Die
Theorie macht Vorhersagen, die im Experiment überprüft
werden. Umgekehrt kann im Experiment ein neuer Effekt beobachtet
werden, für den die Theorie eine Erklärung liefert. Durch den Einsatz
von Computern ist dieses Bild komplizierter geworden. In der folgenden
Graphik sind die Bereiche farblich hinterlegt, in denen heutzutage
Computer zum Einsatz kommen, die hellroten Bereiche werden in dieser
Vorlesung behandelt:
\begin{center}
  \tikzset{>=stealth',
    master/.style={rectangle,very thick,
      rounded corners,draw=black,font={\sffamily\bfseries\Large},
      node distance=6em and 0em},
    subnode/.style={rectangle,very thick,
      draw=black,fill=gray!10!white,font={\sffamily},
      node distance=0.2em},
    edge/.style={very thick},
    vorl/.style={fill=red!20!white}
  }
  \begin{tikzpicture}
    \node[master,vorl] (sim) {Simulation};
    \node[master,above left= of sim] (theo) {Theorie};
    \node[master,above right= of sim] (exp) {Experiment};
    \node[subnode,vorl,above left= of theo] (sym) {Symbolische Mathematik};
    \node[subnode,vorl,below left= of theo] (num) {Numerische Lösung};
    \node[subnode,vorl,above right= of exp] (ausw) {Auswertung};
    \node[subnode,below right= of exp,draw=gray] (mess) {Steuerung};

    \draw[edge] (theo) -- (sym);
    \draw[edge] (theo) -- (num);
    \draw[edge] (exp)  -- (ausw);
    \draw[edge] (exp)  -- (mess);

    \draw[edge,->] (theo) to [bend left] node[above] {Vorhersage} (exp);
    \draw[edge,->] (exp) to [bend left] node[below] {Überprüfung} (theo);

    \draw[edge,->] (exp.south) to [bend left=30] node[left] {Eichung} (sim.east);
    \draw[edge,->] (sim.east) to [bend right=60] node[right] {Vorhersage} (exp.south);

    \draw[edge,->] (theo.south) to [bend right=30] node[right] {Vorhersage} (sim.west);
    \draw[edge,->] (sim.west) to [bend left=60] node[left] {Überprüfung} (theo.south);
  \end{tikzpicture}
\end{center}
Zu den klassischen Säulen Theorie und Experiment ist die
\emph{Simulation} als Mittelding zwischen Theorie und Experiment
gekommen. Computersimulationen stellen Experimente im Computer nach,
ausgehend von bekannten theoretischen Grundlagen. Praktisch alles kann
simuliert werden, von Galaxien bis hin zu Elektronen und Quarks. Dazu
gibt es eine Vielzahl an unterschiedlichen Methoden. Simulationen
erfüllen zwei Hauptaufgaben: Simulationen können einerseits
Experimente ziemlich genau reproduzieren, andererseits kann man mit
Ihrer Hilfe theoretische Modelle in ihrer vollen Komplexität
untersuchen.

Simulationen, die an ein Experiment angepasst (geeicht) sind, können
zusätzliche Informationen liefern, die experimentell nicht zugänglich
sind. Zum Beispiel kann man dort Energiebeiträge getrennt messen oder
sehr kurzlebige Zwischenprodukte beobachten. Außerdem erlauben
Simulationen, Wechselwirkungen und andere Parameter gezielt zu
verändern, und damit Vorhersagen über zukünftige Experimente zu
machen.

Simulationen, die auf theoretischen Modellen basieren, sind oft ein
gutes Mittel, um notwendige Näherungen auf Plausibilität zu überprüfen
oder um einen ersten Eindruck vom Verhalten dieses Modells zu
erhalten. Damit können Simulationen auch helfen, zu entscheiden, ob
notwendige Näherungen oder das Modell unvollständig ist, wenn Theorie
und Experiment nicht zu einander passen.

In der klassischen theoretischen Physik werden Papier und Bleistift
zunehmend vom Computer verdrängt, denn \emph{Computeralgebra} ist
mittlerweile sehr leistungsfähig und kann zum Beispiel in wenigen
Sekunden komplexe Integrale analytisch lösen. Und falls eine Gleichung
doch einmal zu kompliziert ist für eine analytische Lösung, so kann
der Computer mit \emph{numerischen Verfahren} oft sehr gute Näherungen
finden.

In der experimentellen Physik fallen immer größere Datenmengen an. Der
LHC erzeugt zum Beispiel pro Jahr etwa 10 Petabyte an Daten, also etwa
200 Millionen DVDs, was über mehrere Rechenzentren verteilt
gespeichert und ausgewertet werden muss. Klar ist, dass nur Computer
diese gigantischen Datenmengen durchforsten können. Aber auch bei
einfacheren Experimenten helfen Computer bei der \emph{Auswertung und
  Aufbereitung} der Daten, zum Beispiel durch Filtern oder
statistische Analysen. Viele Experimente, nicht nur der LHC, sind aber
auch so komplex, dass Computer zur \emph{Steuerung} der Experimente
benötigt werden, was wir in dieser Vorlesung aber nicht behandeln
können. Die Auswertung und Aufbereitung der Daten hingegen wird
besprochen, auch weil dies genauso auch für Computersimulationen
benutzt wird.

Neben diesen direkten Anwendungen in der Physik ist der Computer
mittlerweile natürlich auch ein wichtiges Mittel für den
Wissensaustausch unter Physikern. Quasi alle wissenschaftlichen
Arbeiten, wie etwa dieses Skript, werden heute nicht mit der
Schreibmaschine und Schablonen erzeugt, sondern auf dem Computer. Die
großen Verlage verlangen mittlerweile auch, Manuskripte als
elektronische Dokumente zur Publikation einzureichen. Umgekehrt stehen
wissenschaftliche Arbeit, vor allem Zeitschriftentexte, normalerweise
nur noch in elektronischen Bibliotheken zur Verfügung, dafür aber
Texte aus der gesamten Welt. Zur Suche in diesen riesigen Datenmengen
dienen wiederum Computer. Und schließlich ist der Computer natürlich
auch unverzichtbar, um international zusammenzuarbeiten - Brief und
Telefon wären schlicht zu langsam und unflexibel. Diese Aspekte wurden
aber schon in den Computergrundlagen behandelt, und sind nicht Teil
dieser Vorlesung.

Um den Computer für Simulationen, Auswertung von Daten oder auch
Lösung komplexer Differenzialgleichungen nutzen zu können, sind neben
physikalischen Kenntnissen auch solche in Programmierung, numerischer
Mathematik und Informatik gefragt. In diesem Skript geht es vor allem
um die grundlegenden Methoden und wie diese angewandt werden, daher
dominiert die numerische Mathematik etwas. Anders als in einer
richtigen Vorlesung zur Numerik stehen hier aber die Methoden und
Anwendungen anstatt der Herleitungen im Vordergrund.

\section{Über dieses Skript}

Im folgenden wird eine in der numerischen Mathematik übliche Notation
benutzt. Wie auch in den meisten Programmiersprachen werden skalare
und vektorielle Variablen nicht durch ihre Schreibweise unterschieden,
allerdings werden üblicherweise die Namen $i$--$l$ für (ganzzahlige)
Schleifenindizes benutzt, $n$ und $m$ für Dimensionen. Da Schleifen sehr
häufig auftreten, wird hierfür die Kurznotation Anfang(Inkrement)Ende
benutzt. Zum Beispiel bedeuten
\begin{align*}
  1(1)n\,=\,1,2,\ldots, n\\
  n(-2)1\,=\,n, n-2,\ldots, 3, 1.
\end{align*}
Alle anderen Variablen sind reellwertige Skalare oder Vektoren,
$\RR^n$ bezeichnet dabei den $n$-dimensionalen Vektorraum reller
Zahlen. Mit $e_i$ wird dabei der Einheitsvektor der $i$-ten Spalte
bezeichnet, mit $e_i^T$ seine Transponierte, also der Einheitsvektor
der $i$-ten Zeile. Bei komplexen Vektoren steht entsprechend der obere
Index $^H$ für die Hermitesche eines Vektors, also die Transponierte
des Vektors, bei der alle Komponenten komplex konjugiert sind. Für
reelle Vektoren sind also $^H$ und $^T$ gleichbedeutend.

Integrale werden mit dem Volumenelement am Ende geschrieben, dessen
Dimensionalität sich aus dem Integrationsbereich erschließt. Sehr
häufig werden Abschätzungen mit Hilfe der \emph{Landau}-Symbole
verkürzt. Wie üblich heißt für zwei Funktionen $f$ und $g$
\begin{equation*}
  f = \O_{x\to a}(g) \quad\iff\quad \lim_{x\to a}
  \frac{\lvert f(x)\rvert}{\lvert g(x)\rvert}<\infty.
\end{equation*}
In den meisten Fällen ist $a=0$ oder $a=\infty$ und aus dem Kontext
klar, welcher Grenzwert gemeint ist. Dann wird die Angabe
weggelassen. Oft wird auch die Notation $f = g + \O(h)$ benutzt, um $f
+ g = \O(h)$ auszudrücken. $f(x) \dot{=} g(x)$ schließlich bedeutet 
$f(x) - g(x) = \O{x\to 0}(x)$.

Um einzelne Methoden konkret vorstellen zu können, wird in diesem
Skript auf die Sprachen Python und C zurückgegriffen. Im Bereich des
Hochleistungsrechnens werden vor allem die Sprachen Fortran und C/C++
eingesetzt, weil diese in Verbindung mit guten Compilern sehr
effizienten Code ergeben. Allerdings bieten diese Sprachen keine
nativen Datentypen wie zum Bespiel Listen oder Wörterbücher und
verlangen die explizite Typisierung von Variablen, was Beispiele
unnötig verkompliziert. Daher benutzt dieses Skript die
Programmiersprache Python\footnote{\url{www.python.org}} mit den
Erweiterungen NumPy und SciPy\footnote{\url{http://www.scipy.org}},
die eine leistungsfähige numerische Bibliothek und umfangreiche
Visualisierungsmöglichkeiten bietet. Für elementare Beispiele hingegen
greift dieses Skript auf das hardwarenähere C zurück.

Zum Erlernen der Programmiersprachen Python und C sei auf die
Materialien der Veranstaltung „Computergrundlagen“ hingewiesen, die im
Fachbereich Physik der Universität Stuttgart jährlich angeboten
wird.

Die meisten der in diesem Skript vorgestellten numerischen Methoden
werden von SciPy direkt unterstützt. Die Qualität dieser
Implementationen ist mit eigenem Code nur schwierig zu überbieten. Wo
immer möglich, wird daher auf die entsprechend SciPy-Befehle
verwiesen. Zum Beispiel wird auf die Funktion "`method"' im
SciPy-Modul "`library"' in der in der Form
\scipy{scipy.library.method(arg1, arg2, ...)} verwiesen. Trotzdem
sind viele Methoden auch als expliziter Code gezeigt, da man natürlich
eine Vorstellung davon haben sollte, was diese Methoden tun.

\section{Beispiel: Fadenpendel}
\index{Fadenpendel}

\begin{figure}
  \centering
  \newcommand{\auslenkung}{50}
  \newcommand{\kraft}{1.5}
  \begin{tikzpicture}[x=3em,y=3em]
    \draw[dashed] (0,1)--(0,-2.5);

    % Winkel
    \filldraw[fill=none,draw=black] (0,-1.5) arc (-90:-90-\auslenkung:1.5);
    \pgftransformrotate{-\auslenkung/2}
    \node at (-0.0,-1) {$\alpha$};
    \pgftransformrotate{\auslenkung/2}

    % ausgelenkte Kugel
    \pgftransformrotate{-\auslenkung}

    \draw[color=black,dashed] (0,1) -- (0,-3);
    \draw[thick,fill] (0,-3) circle (0.18cm) node (masse) {};
    \draw[thick] (0,0) circle (.1cm);
    \draw[thick] (0,0) -- (0,-3);
    \draw[decorate,decoration=brace] (-.2,-3) -- (-.2,0);
    \node at (-0.7,-1.5) {$l$};

    \draw[->,color=blue,thick] (masse) -- +({\kraft*sin(\auslenkung)},0)
    node (ft) {} node[right] {$F_\text{tan} = F_\text{G}\sin(\alpha)$};
    \draw[->,color=red,thick] (masse) -- +(0,{-\kraft*cos(\auslenkung)})
    node (fr) {} node[left] {$F_\text{rad}$};

    % totale Kraft
    \pgftransformrotate{\auslenkung}
    \draw[->,very thick] (masse) -- +(0,-\kraft)
    node (fg) {} node[below] {$F_\text{G} = -mg$};
    \draw[thin,dotted] (fr.center) -- (fg.center) -- (ft.center);
  \end{tikzpicture}
  \caption{Schematisches Fadenpendel der Masse $m$, das an einem
    masselosen, steifen Faden der Länge $l$ hängt.}
  \label{fig:pendel}
\end{figure}

Wir betrachten ein einfaches Beispielsystem, nämlich ein Fadenpendel.
Wird dieses Pendel nun ausgelenkt, vollführt es eine periodische
Schwingung um die Ruhelage, d.h., den tiefsten Punkt. Unser Ziel als
Physiker ist nun, die Position der Kugel als Funktion der Zeit
vorherzusagen. Das allerdings ist eine unmögliche Aufgabe --- man
stelle sich zum Beispiel eine stark inhomogene Masse vor (oder ein
Fadenpendel als Masse) oder dass der Faden elastisch ist. Daher müssen
wir zunächst ein geeignet vereinfachtes \emph{Modell} erstellen, auf
das wir dann die bekannten physikalischen Gesetze anwenden
können.

\subsection{Modell}

Als Modell wählen wir eine homogene Kugel der Masse $m$, die an einem
masselosen, steifen Faden der Länge $l$ hängt (vergleiche
Figur~\ref{fig:pendel}). Auf diese Kugel wirkt nur eine Gewichtskraft
der Größe $mg$ senkrecht nach, alle anderen Kräfte vernachlässigen wir
komplett, insbesondere auch die Reibung.

Da der Faden unendlich steif sein soll, kann sich Kugel lediglich auf
einem Kreis mit Radius $l$ um die Aufhängung bewegen, d.h. die
Position der Kugel ist durch die Auslenkung $\alpha$ aus dem tiefsten
Punkt vollständig beschrieben. Weiter wird die Komponente der Kraft
parallel zum Faden komplett von diesem kompensiert, daher bleibt bei
Auslenkung $\alpha$ von der Gewichtskraft nur ihre Komponente
\begin{equation}
  F_\text{tan} = F_\text{G}\sin(\alpha) = -mg\sin(\alpha)
\end{equation}
senkrecht zum Faden übrig. Das Newtonsche Gesetz besagt nun, dass
die Tangentialbeschleunigung, also die Beschleunigung entlang $\alpha$
\begin{equation}
  l\ddot\alpha = F_\text{tan}/m = -g\sin(\alpha)
  \label{eq:pendelgln}
\end{equation}
beträgt. Dies ist jetzt eine Differentialgleichung für die Auslenkung
$\alpha(t)$ des Pendels als Funktion der Zeit. Diese wiederum liefert
uns die gewünschte Position $(\cos(\alpha)l,\sin(\alpha)l)$ der Kugel
relativ zur Aufhängung als Funktion der Zeit. Leider hat selbst diese
einfache Differentialgleichung keine geschlossene Lösung, und wir
müssen weitere Näherungen einführen, um eine analytische Lösung zu
erhalten.

\subsection{Näherung: der harmonische Oszillator}

\begin{figure}
  \centering
  \includegraphics[width=\textwidth]{plots/pendel_loesung}
  \caption{Lösungen für ein Fadenpendel der Länge $l=1m$. Im linken Graphen
    ist die Ausgangslage $\alpha=1,5$, im rechten $\alpha=0,1$; in
    beiden Fällen ist die Ausgangsgeschwindigkeit 0. Die durchgezogene
  grüne Linie markiert die analytische Näherungslösung
  \eqref{eq:naeherung} für kleine Winkel. $\times$ markiert die
  Ergebnisse einer Integration mit dem einfachen Vorwärtsschritt
  \eqref{eq:simple} mit Zeitschritt $0,1s$, die gestrichelte rote
  Linie mit Zeitschritt $0,01s$. $+$ markiert die Lösung mit Hilfe des
  Velocity-Verlet-Algorithmus und Zeitschritt $0,1s$.}
  \label{fig:loesung}
\end{figure}

Für kleine Winkel gilt $\sin(\alpha)\approx\alpha$, und damit
\begin{equation}
  \ddot\alpha \approx = -\frac{g}{l}\alpha.
\end{equation}
Diese Differentialgleichung hat die allgemeine Lösung
\begin{equation}
  \alpha(t) = A \sin(\omega t + \phi)
  \label{eq:naeherung}
\end{equation}
mit $\omega=\sqrt{g/l}$, wie man sich leicht durch Einsetzen
überzeugt. Die Größen $A$ und $\phi$ ergeben sich aus den
Anfangsbedingungen, nämlich der Anfangsposition
\begin{equation}
  \alpha_0 = A \sin(\phi)
\end{equation}
und "~geschwindigkeit
\begin{equation}
  v_0 = A \omega \cos(\phi).
\end{equation}
Ist zum Beispiel $v_0=0$, so ist $\phi=\pi/2$ und $A=\alpha_0$, im
allgemeinen Fall ist
\begin{equation}
\phi =
\arctan\left(\frac{\alpha_0\omega}{v_0}\right)\quad\text{und}\quad
A = \frac{\alpha_0}{\sin(\phi)}.
\end{equation}
Wir haben nun eine geschlossene Lösung für die Position des Pendels,
so lange die Ausgangslage nicht zu sehr ausgelenkt ist. Um diese
Lösung zu visualisieren, nutzt man heute üblicherweise den Computer,
siehe Graph~\ref{fig:loesung}.

\subsection{Numerische Lösung}

Was passiert nun, wenn das System stärker ausgelenkt ist? Mit sehr
viel mehr Aufwand lässt sich auch für diesen Fall eine analytische
Lösung finden, allerdings in Form einer Reihe, die nicht mehr so
einfach zu zeichnen ist. Eine Alternative ist, die
Differentialgleichung \eqref{eq:pendelgln} mit Hilfe des Computers zu
berechnen. Wir sagen, wir "`simulieren"' das Pendel. Dazu fixieren wir
ein Einheitensystem, zum Beispiel eine Sekunde als Zeiteinheit und
einen Meter als Längeneinheit. In diesem System ist also $l=1$,
$g\approx 9,81$ und $\omega\approx 3,13$, falls des Pendel einen Meter
lang ist.

Zunächst müssen wir das Problem aber für den Computer anpassen, der ja
nur mit (endlich vielen) gewöhnlichen Zahlen rechnen kann, wir müssen
das Problem \emph{diskretisieren}. Wir betrachten nur die Zeitpunkte
\begin{equation}
  t_n = n\delta t, n=1(1)N,
\end{equation}
wobei der Zeitschritt $\delta t$ frei wählbar ist. Je kleiner $\delta
t$, desto genauer können wir $\alpha(t)$ bestimmen, allerdings steigt
natürlich die Anzahl der Schritte, die nötig sind, um eine feste
Gesamtzeit zu erreichen. Unsere Lösung, die Funktion $\alpha(t)$ wird
also durch ihre Werte $\alpha(t_n)$ an den diskreten Zeitpunkten
dargestellt.

\begin{figure}
  \centering
  \includegraphics[width=\textwidth]{plots/pendel_energie}
  \caption{Energie als Funktion der Zeit, wieder für $l=1m$, und
    Ausgangslage $\alpha=1,5$ (links) und $\alpha=0,1$ (rechts) in
    Ruhe.  $+$ markiert die Ergebnisse einer Integration mit dem
    einfachen Vorwärtsschritt \eqref{eq:simple} mit Zeitschritt
    $0,1s$, die gestrichelte rote Linie mit Zeitschritt $0,01s$. $\times$
    markiert die Lösung mit Hilfe des Velocity-Verlet-Algorithmus und
    Zeitschritt $0,1s$, und die gestrichelte blaue Linie mit $0,01s$.}
  \label{fig:energie}
\end{figure}

Um Gleichung \eqref{eq:pendelgln} auf den Computer zu bringen, müssen
wir uns allerdings noch überlegen, wie wir mit der Ableitung verfahren.
Da wir die Ausgangsposition und "~ge\-schwin\-dig\-keit gegeben haben, liegt
es nahe, die Gleichung zu integrieren:
\begin{equation}
  v(t+\delta t) = \dot\alpha(t + \delta t) = v(t) + \int_{t}^{t+\delta t}
  -\omega^2\sin \alpha(\tau)\, d\tau.
\end{equation}
Da $\delta t$ aber unser Zeitschritt ist, wir also nichts weiter über
$\alpha(\tau)$ wissen, bietet sich die folgende Näherung an:
\begin{equation}
  v(t+\delta t) \approx v(t) - \omega^2\sin\alpha(t) \delta t.
\end{equation}
Analog ergibt sich dann durch nochmalige Integration:
\begin{equation}
  \alpha(t+\delta t) \approx \alpha(t) + v(t) \delta t.
  \label{eq:simple}
\end{equation}
Ausgehend von
\begin{equation}
  \alpha(0) = \alpha_0 \quad\text{und}\quad v(0) = v_0
\end{equation}
lässt sich damit also $\alpha(t)$ numerisch bestimmen. Der
Quellcode~\ref{lst:pendel} zeigt, wie eine einfach Implementation in
Python aussehen könnte.

Wie kann man nun überprüfen, ob diese Lösung tatsächlich korrekt ist?
Da das System abgeschlossen ist, muss seine Energie
\begin{equation}
  E = \frac{1}{2} l^2 v(t)^2 + gl (1 - \cos(\alpha(t)))
\end{equation}
erhalten sein. Lässt man sich diese allerdings ausgeben, stellt man
fest, dass $E(t)$ erheblich schwankt, vergleiche
Graph~\ref{fig:energie}. Dies lässt sich durch Verringern des
Zeitschritts beheben, das kostet aber entsprechend mehr Rechenzeit.

Eine bessere Alternative ist, den Algorithmus zu verbessern, was
wiederum etwas analytische Arbeit erfordert. Wir betrachten die
Taylorentwicklungen
\begin{align}
  \alpha\left(t+\delta t\right) &= \alpha\left(t+\frac{\delta t}{2}\right) + \frac{\delta t}{2}
  v\left(t+\frac{\delta t}{2}\right) + \frac{\delta t^2}{8} F\left(t+\frac{\delta t}{2}\right) + \O(\delta t^3)
  \intertext{und}
  \alpha\left(t\right) &= \alpha\left(t+\frac{\delta t}{2}\right) - \frac{\delta t}{2}
  v\left(t+\frac{\delta t}{2}\right) + \frac{\delta t^2}{8} F\left(t+\frac{\delta t}{2}\right) - \O(\delta t^3).
\end{align}
Durch Subtraktion ergibt sich
\begin{align}
  \alpha(t+\delta t) = \alpha(t) + \delta t\,v\left(t+\frac{\delta t}{2}\right) + \O(\delta t^4).
\end{align}
Die Geschwindigkeiten an den halben Zeitschritten erhält man einfach
durch $v(t + \delta t /2) = v(t) + \delta t F(t) / 2$.
Zusammengefasst ergibt sich der folgende \emph{Velocity-Verlet-Algorithmus}:
\begin{align}
  v\left(t + \frac{\delta t}{2}\right) &= v(t) + \frac{\delta t}{2} F(t) \\
  \alpha(t + \delta t) &= \alpha(t) + v\left(t + \frac{\delta t}{2}\right) \\
  v(t + \delta t) &= v\left(t + \frac{\delta t}{2}\right) + \frac{\delta t}{2}
  F(t + \delta t),
\end{align}
der anders als die direkte Vorgehensweise vorher numerisch stabil ist
und quasi keine Energieschwankungen aufzeigt, vergleiche
Graph~\ref{fig:energie}. 
Im Quellcode~\ref{lst:pendel} ist alternativ auch dieser Integrator
implementiert. Obwohl er nur unwesentlich komplizierter ist als der
einfache Integrator zuvor, erreicht etwa dieselbe Genauigkeit wie
dieser mit einem Zehntel der Zeitschritte.

Als weiterer Test bietet sich an, bei kleinen Auslenkungen mit der
analytisch bekannten Lösung zu vergleichen, die gut reproduziert wird,
siehe Graph~\ref{fig:loesung}. Bei größeren Anfangsauslenkungen oder
-ge\-schwin\-dig\-keiten ist die Abweichung allerdings sehr groß, weil
hier die analytische Näherung versagt. Im Rahmen ihrer Genauigkeit
erlaubt also die numerische Lösung, das vorgegebene Modell in einem
größeren Parameterraum auf sein Verhalten hin zu untersuchen, als
analytisch möglich wäre.

\lstinputlisting[style=floating,
caption={[Fadenpendel] Python-Code zum Fadenpendel
  mit graphisch aufbereiteter Ausgabe mit Hilfe der \texttt{matplotlib}.},
label=lst:pendel]{pendel.py}

%%% Local Variables: 
%%% mode: latex
%%% TeX-master: "padc"
%%% TeX-PDF-mode: t
%%% End: 
